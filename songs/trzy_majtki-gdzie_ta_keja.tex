\newpage
\begin{song}{title={Gdzie ta keja}, music={Jerzy Porębski}, interpret={Trzy Majtki}}
    \begin{verse}
        Gdyby ^{a}tak ktoś przyszedł i powiedział: ``S^{E}tary, czy masz ^{a}czas? \\
        Potrze^{C}buję do załogi j^{G}akąś nową t^{C}warz \\
        Ama^{C}zonka, Wielka R^{C7}afa, ^{d} oceany trzy \\
        Rejs na ^{a}całość, rok, dwa lata'' --- ^{E}to powiedział^{a}bym:
    \end{verse}
  	\begin{chorus}
        Gdzie ta ^{a}keja, a ^{E}przy niej ten j^{a}acht \\
        Gdzie ta ^{C}koja, ^{G}wymarzona w s^{C}nach \\
        Gdzie te w^{g}szystkie sz^{A7}nurki ^{d}od ^{A7}tych sz^{d}mat \\
        Gdzie ta b^{a}rama ^{E}na szeroki ś^{a}wiat
    \end{chorus}
    \begin{chorus*}
        Gdzie ta keja, a przy niej ten jacht \\
        Gdzie ta koja wymarzona w snach \\
        W każdej chwili płynę w taki rejs \\
        Tylko gdzie to jest, no gdzie to jest
    \end{chorus*}
    \begin{verse}
        Gdzieś na dnie wielkiej szafy leży ostry nóż \\
        Stare jeansy wystrzępione impregnuje kurz \\
        W kompasie igła zardzewiała, lecz kierunek znam \\
        Biorę wór na plecy i przed siebie gnam
    \end{verse}
  	\begin{chorus}
        Gdzie ta keja, a przy niej ten jacht\ldots
    \end{chorus}
    \begin{verse}
        Przeszły lata zapyziałe, rzęsą zarósł staw \\
        A na przystani czółno stało --- kolorowy paw \\
        Zaokrągliły się marzenia, wyjałowiał step \\
        Lecz dalej marzy o załodze ten samotny łeb
    \end{verse}
  	\begin{chorus}
        Gdzie ta keja, a przy niej ten jacht\ldots
    \end{chorus}
\end{song}

