\newpage
\begin{song}{title={Byłaś serca biciem}, music={Jerzy Dobrzyński}, interpret={Andrzej Zaucha}, capo=3, annex}
    \begin{intro}
    \writechord{a} \writechord{G} \writechord{d7} \writechord{d7}
    \end{intro}
    \begin{chorus}
        ^{a}Byłaś serca b^{G}iciem ^{d7} \\
        ^{a}Wiosną, zimą, ż^{G}yciem ^{d7} \\
        ^{a}Marzeń moich e^{G}chem ^{d7} \\
        ^{a}Winem, wiatrem, śmi^{G}echem ^{d7}
    \end{chorus}
    \begin{verse}
        Ostatnio w ^{a}mieście mym tramwaje ^{G}po północy b^{d7}łądzą \\
        Rozkładem n^{a}ocnych tras piekielne ^{G}jakieś moce rz^{d7}ądzą \\
        Nie wiedzieć ^{a}czemu wciąż rozkł^{G}ady jazdy ta^{d7}k zmieniają \\
        Że prawie k^{a}ażdy tramwaj ^{G}pod twym oknem ^{d7}nocą staje
    \end{verse}
      \begin{chorus}
        Byłaś serca biciem\ldots
    \end{chorus}
    \begin{verse}
        Ostatnio słońca mniej, ostatnio noce bardziej ciemne \\
        Już nawet księżyc drań o tobie nie chce gadać ze mną \\
        W kieszeni grosze dwa, w kieszeni na dwa szczęścia grosze \\
        W tym jednak losu żart, że ja obydwa grosze noszę
    \end{verse}
    \begin{chorus}
        Byłaś serca biciem\ldots
    \end{chorus}
    \begin{interlude}
        ^{F}Ktoś p^{d7}ytał jak się m^{C}asz, j^{a}ak się czujesz \\
        ^{F}Ktoś, z k^{d7}im rok w wojnę g^{C}rasz wycze^*{e}ku ^{A}je \\
        ^{F}Ktoś, k^{d7}to nocami, u^{C}licami, t^{a}ramwajami \\
        ^{F}Pod twe okno ^{d7}mknie, gdzie spo^*{D}ty ^{Eb}ka ^{E}mnie
    \end{interlude}
    \begin{chorus}
        Byłaś serca biciem\ldots
    \end{chorus}
    \begin{interlude}
        Ktoś pytał jak się masz\ldots
    \end{interlude}
\end{song}

