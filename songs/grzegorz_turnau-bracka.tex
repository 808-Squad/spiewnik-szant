\newpage
\begin{song}{title={Bracka}, music={Grzegorz Turnau}, lyrics={Michał Zabłocki}}
\begin{multicols}{2}
    \begin{intro}
    \writechord{g#7} \writechord{d#7} \writechord{H7/A} \writechord{E/G#} \\
    \writechord{G} \writechord{D/F#} \writechord{a6} \\
    \writechord{D7/C} \writechord{G7/F} \writechord{H7/A} \writechord{e/G} \\
    \writechord{G7/F} \writechord{E} \\ \\
    \writechord{D/C} \writechord{Eadd9} \\
    \end{intro}
    \begin{verse}
        Na półn^{g#7}ocy ściął mr^{d#7}óz \\
        Z nieba spa^{H7/A}dł wielki w^{E/G#}óz \\
        Przykrył dr^{G}ogi po^{D/F#}la i fal^{a6}e \\
        Myśli zm^{D7/C}arzły na ló^{G7/F}d \\
        Dobre s^{H7/A}ny zmorzył gł^{e/G}ód \\
        Lecz przy^{G7/F}najmniej się można przes^{E}traszyć 
    \end{verse}
    \begin{verse}
        Na południu już skwar \\
        Miękki puch z nieba zdarł \\
        Kruchy pejzaż na piasek przepalił \\
        Jak upalnie mój Boże \\
        Lecz przynajmniej być może \\
        Wreszcie byśmy się tam zakochali 
    \end{verse}
    \begin{interlude}
        \writechord{a} \writechord{G} \writechord{d} \writechord{e} $\times 3$ \\
        \writechord{F} \writechord{G7/F} \writechord{d} \writechord{B} \\
        \writechord{a} \writechord{G} \writechord{d} \writechord{e} \\
    \end{interlude}
    \begin{chorus}
        ^{a}A w Krak^{G}owie na Bra^{d}ckiej pada de^{e}szcz \\
        Gdy koni^{F}eczność istn^{G/F}ienia \\
        Tudna je^{d}st do znies^{B}ienia \\
        ^{a}W koryt^{G}arzu i w ku^{d}chni pada t^{e}eż \\
        Przykle^{F}jony do ści^{G}any zwijam mo^{d}kre dyw^{B}any \\
        Nie od de^{a}szczu m^{G}okre, lecz od ł^{d}ez -^{E}-- \\ \\
        \writechord{D/C} \writechord{Eadd9} 
    \end{chorus}
    \begin{verse}
        Na zachodzie już noc \\
        Wciągasz głowę pod koc \\
        Raz zasypiasz i sprawa jest czysta \\
        Dłonie zapleć i złóż \\
        Nie obudzisz się już \\
        Lecz przynajmniej raz możesz się wyspać 
    \end{verse}   
    \begin{verse}
        Jeśli wrażeń Cię głód \\
        Zagna kiedyś na wschód \\
        Nie za długo tam chyba wytrzymasz \\
        Lecz na wschodzie przynajmniej życie płynie zwyczajnie \\
        Słońce wschodzi i dzień się zaczyna 
    \end{verse}
    \begin{chorus}
        A w Krakowie na Brackiej pada deszcz \\
        Przemęczony i senny zlew przecieka kuchenny \\
        Kaloryfer jak mysz się poci też \\
        Z góry na dół kałuże przepływają po sznurze \\
        Nie od deszczu mokrym lecz od łez \\ \\
        Bo w Krakowie na Brackiej pada deszcz \\
        Gdy zagadka istnienia \\
        Zmusza mnie do myślenia \\
        W korytarzu i w kuchni pada też \\
        Przyklejony do ściany zwijam mokre dywany \\
        Nie od deszczu mokre lecz od łez \\ \\
        \writechord{D/C} \writechord{Eadd9} 
    \end{chorus}
    \begin{outro}
        ^{a}Bo w Krak^{G}owie na Br^{d}ackiej pada d^{e}eszcz \\
        ^{a}Bo w Krak^{G}owie na Br^{d}ackiej pada -^{e}-- \\
        Pada de^{F}szcz --^{G}- \\
        Pada de^{a}szcz
    \end{outro}
\end{multicols}
\end{song}

