\newpage
\begin{song}{title={Nazywali go Marynarz (Szanta narciarska)}, music={Artur Andrus}}
	\begin{intro}
	\writechord{d}
	\end{intro}    
    \begin{multicols}{2}
    \begin{verse}
        Nazy^{d}wali ^{C}go ma^{d}rynarz \\
        Bo o^{d}paskę ^{C}miał na ^{d}oku \\
        ^{g}Na każdym stoku dziew^{d}czyna \\
        Dziew^{B}czyna na ^{A}każdym s^{d}toku
    \end{verse}
    \begin{verse*}
        Po^{d}chodzi s^{C}pod Poz^{d}nania \\
        Po^{d}dobno ^{C}umie w^{F}różyć z kart \\
        ^{g}Panny rwie na wią^{d}zania \\
        Mę^{B}żatki --- ^{A}na długość ^{d}nart
    \end{verse*}
  	\begin{chorus}
        Ca^{d}ryco ^{A}mokrego ś^{d}niegu \\
        Ra^{d}trakiem płynę do ciebie pod ^{g}prąd --- hej! \\
        ^{g}Dobrze, że stoisz na ^{d}brzegu \\
        Bo ja ^{B}właśnie ^{A}schodzę na ^{d}ląd
    \end{chorus}
        \vfill\null\columnbreak{}
    \begin{verse}
        Nigdy się nie lękał biedy \\
        I się nie przejmował jutrem \\
        A jego ratrak był kiedyś \\
        Zwyczajnym, rybackim kutrem
    \end{verse}
    \begin{verse*}
        I woził dorsze i śledzie \\
        Zimą i latem, okrągły rok \\
        Teraz, jak nieraz przejedzie \\
        Rybami --- czuć cały stok
    \end{verse*}
    \begin{chorus}
        Caryco mokrego śniegu\ldots
    \end{chorus}
    \begin{verse}
        Wszyscy w porcie odetchnęli \\
        Zwiał, nim się zakończył sezon \\
        Jeszcze się tam, jak żagiel bieli \\
        Jego czarny kombinezon
    \end{verse}
    \begin{verse*}
        Odpłynął pod Ustrzyki \\
        I przez kobiety wpadł w kłopoty \\
        Forsę z polowań na orczyki \\
        Przehulał --- na antybiotyk
    \end{verse*}
    \begin{chorus}
        Caryco mokrego śniegu\ldots
    \end{chorus}
    \begin{verse}
        Jeśli kiedyś go zobaczysz \\
        Na ratraku w podłym świecie \\
        To powiedz mu, że w Karpaczu \\
        Czekają na niego dzieci
    \end{verse}
    \begin{verse*}
        I kiedy opuszcza statek \\
        Żeby się znowu oddać złu \\
        Każda z dwudziestu siedmiu matek \\
        Dzieciątku --- śpiewa do snu
    \end{verse*}
    \begin{chorus}
        Caryco mokrego śniegu\ldots $\times 2$
    \end{chorus}
    \end{multicols}
\end{song}

