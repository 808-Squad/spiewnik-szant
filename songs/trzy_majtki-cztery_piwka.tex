\newpage
\begin{song}{title={Cztery piwka}, interpret={Trzy Majtki}, music={Jerzy Porębski}}
    \small
    \begin{verse}
        ^{g}Ze Świnoujścia do Walvis Bay droga nie była krótka \\
        A ^{g}po dwóch dobach (albo mniej) j^{D}uż się skończyła w^{g}ódka \\
        \say{Do b^{g}rydża!} --- krzyknął Siwy Flak i z miejsca rzekł: \say{Dwa piki} \\
        A ^{g}ochmistrz w \say{telewizor} wlał nie ^{D}byle jakie ^{G}siki
    \end{verse}
    \begin{chorus}
        Cztery ^{G}piwka na stół, w ^{C}popielniczkę pet \\
        J^{D}akąś damę roześmianą król przytuli wn^{G}et \\
        G^{G}dzieś między palcami ^{C}sennie płynie czas \\
        Cz^{D}warta ręka króla bije as ^{g}
    \end{chorus}
    \begin{verse}
        A w karcie tylko jeden as i nic poza tym nie ma \\
        Ale nie powiem przecie \say{pas}, może zagrają szlema \\
        \say{Kontra} --- mu rzekłem, taki blef, by nieco spuścił z tonu \\
        A Fred mi na to: \say{Cztery trefl!} --- przywalił bez pardonu
    \end{verse}
    \begin{chorus}
        Cztery piwka na stół, w popielniczkę pet\ldots
    \end{chorus}
    \begin{verse}
        A mój w dwa palce obtarł nos, to znaczy: \say{Nie mam nic} \\
        I wtedy Flak, podnosząc głos, powiedział: \say{Cztery pik!} \\
        I kiedy jeszcze cztery króle pokazał mu jak trza \\
        To Fred z renonsem: \say{Siedem pik!} powiedział, \say{niech gra Flak!}
    \end{verse}
    \begin{chorus}
        Cztery piwka na stół, w popielniczkę pet\ldots
    \end{chorus}
    \begin{verse}
        A ja mu \say{kontra}, on mi \say{re}, ja czuję pełen luz \\
        Bo widzę w moich kartach, że jest atutowy tuz \\
        Więc strzelam! Kiedy karty Fred wyłożył mu na blat \\
        To każdy mógł zobaczyć, jak Siwego Flaka trafia szlag
    \end{verse}
    \begin{chorus}
        Cztery piwka na stół, w popielniczkę pet\ldots
    \end{chorus}
    \begin{verse}
        Już nie pamiętam, ile dni w miesiące złożył czas \\
        Morszczuki dosyć dobrze szły i grało się nieraz \\
        Lecz nigdy więcej Siwy Flak, klnę na jumprowe wszy\footnote{potoczna nazwa na odstające druciki w stalowych linach, które powodują swędzące rany na dłoniach} \\
        Choć byś go prosił, tak czy siak, nie zasiadł już do gry
    \end{verse}
    \begin{chorus}
        W popielniczkę pet, cztery piwka na stół \\
        Już tej damy roześmianej nie przytuli król \\
        Gdzieś nam się zapodział atutowy as \\
        Tego szlema z nami wygrał czas
    \end{chorus}
\end{song}

