\newpage
\begin{song}{title={Pejzaże Harasymowiczowskie}, lyrics={Wojciech Bellon}, music={Wolna Grupa Bukowina}}
    \begin{intro}
        \writechord{G} $\times 2$
    \end{intro}
    \begin{verse}
        ^{G} Kiedy wstałem w przed^{D}świcie, a Synaj \\
        ^{C} Prawdę głosił przez ^{e}trąby wiatru \\
        ^{G} Zasmreczyły się ^{D}chmury igliwiem \\
        ^{e} Bure świerki, o ^{C}góry wsp^{D}arte \medskip \\
        I na niebie byłem, ja jeden \\
        Plotąc pieśni w warkocze bukowe \\
        I schodziłem na ziemię za kwestą \\
        Przez skrzydlącą się bramę Lackowej
    \end{verse}
    \begin{chorus}
        ^{G} I był Beskid, i ^{C}były ^{G}słowa \\
        ^{G} Zanurzone po ^{C}pępki w cerkwi \\
        b^{D}aniach, rozło^{D}żyście złotych \\
        ^{C} Smagających się ^{D}wiatrem ^{G}do krwi \medskip \\
        \writechord{G}
    \end{chorus}
    \begin{verse}
        Moje myśli biegały końmi \\
        Po niebieskich, mokrych połoninach --- \\
        I modliłem się, złożywszy dłonie \\
        Do gór, do Madonny Brunatnolicej \smallskip \\
        A gdy serce kroplami tęsknoty \\
        Jęło spadać na góry sine  \\
        Czarodziejskim kwiatem paproci \\
        Rozgwieździła się bukowina
    \end{verse}
    \begin{chorus}
        I był Beskid, i były słowa\ldots
    \end{chorus}
\end{song}

