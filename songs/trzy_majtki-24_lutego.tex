\newpage
\begin{song}{title={24 lutego (Bijatyka)}, interpret={Trzy Majtki}, music={tradycyjna}}
    \begin{verse}
        To dwu^{D}dziesty czwarty był lutego \\
        Po^{A}ranna zrzedła mgła \\
        A ^{h}wyszło z niej siedem uzbrojonych kryp \\
        Tu^{G}recki ^{A}niosły z^{h}nak
    \end{verse}
    \begin{chorus}
        No i z^{D}nów bijatyka, no i znów bijatyka \\
        No i bijatyka cały d^{A}zień \\
        I ^{h}porąbany dzień, i porąbany łeb \\
        Razem ^{G}bracia, ^{A}aż po z^{h}mierzch! $\times 2$
    \end{chorus}
    \begin{verse}
        I już pierwszy zbliża się do burt \\
        A zwie się \say{Goździk Lee} \\
        Z Algieru pasza wysłał go \\
        Żeby nam upuścił krwi
    \end{verse}
    \begin{chorus}
        No i znów bijatyka, no i znów bijatyka\ldots $\times 2$
    \end{chorus}
    \begin{verse}
        Już następny zbliża się do burt \\
        A zwie się \say{Róży Pąk} \\
        Plunęliśmy ze wszystkich luf \\
        Bardzo prędko szedł na dno
    \end{verse}
    \begin{chorus}
        No i znów bijatyka, no i znów bijatyka\ldots $\times 2$
    \end{chorus}
    \begin{verse}
        W naszych rękach dwa, i dwa na dnie \\
        Cała reszta zwiała gdzieś \\
        A jeden z nich zabraliśmy \\
        Aż na Starej Anglii brzeg
    \end{verse}
    \begin{chorus}
        No i znów bijatyka, no i znów bijatyka\ldots $\times 2$
    \end{chorus}
\end{song}

