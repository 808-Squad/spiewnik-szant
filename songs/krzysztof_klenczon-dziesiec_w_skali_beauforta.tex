\newpage
\begin{song}{title={Dziesięć w skali Beauforta}, music={Krzysztof Klenczon}, capo=3}
    \begin{verse}
    	Ko^{e}łysał nas zac^{a}hodni wiatr \\
        ^{H7}Brzeg gdzieś za rufą z^{e}ostał \\
        I n^{a}agle ktoś jak p^{e}apier zbladł \\
        ^{F#7}Sztorm idzie, panie ^{H}bosman
    \end{verse}
    \begin{chorus}
        A ^*{C}bo sm^{G}an tylko ^{C}zapiął pł^{G}aszcz \\
        I z^{C}aklął: - ^{H7}Ech, do cz^{e}orta \\
        Nie da^{C}ję ła^{D}jbie ż^{G}adnych s^{e}zans \\
        ^{e}Dziesięć w ^{a}skali ^*{H7}Beau ^{e}forta
    \end{chorus}
    \begin{verse}
        Z zasłony ołowianych chmur \\
        Ulewa spadła nagle \\
        Rzucało nami w górę, w dół \\
        I fala zmyła żagle
    \end{verse}
    \begin{chorus}
        A bosman tylko zapiął płaszcz\ldots
    \end{chorus}
    \begin{verse}
        Gdzie został ciepły, cichy kąt \\
        I brzegu kształt znajomy \\
        Zasnuły mgły daleki ląd \\
        Dokładnie, z każdej strony
    \end{verse}
    \begin{chorus}
        A bosman tylko zapiął płaszcz\ldots
    \end{chorus}
    \begin{verse}
        O pokład znów uderzył deszcz \\
        I padał już do rana \\
        Piekielnie ciężki to był rejs \\
        Szczególnie dla bosmana
    \end{verse}
    \begin{chorus}
        A bosman tylko zapiął płaszcz \\
        I zaklął: - Ech, do czorta \\
        Przedz^{C}iwne cz^{D}asem s^{G}ny się ^{e}ma \\
        ^{e}Dziesięć w ^{a}skali ^*{H7}Beau ^{e}forta \\
        Dziesięć w skali Beauforta \\
        Dziesięć w skali Beauforta 
     \end{chorus}
\end{song}

