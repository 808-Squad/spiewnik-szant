\newpage
\begin{song}{title={Byłaś serca biciem}, music={Jerzy Dobrzyński}, interpret={Andrzej Zaucha}}
    \begin{intro}
    \writechord{c} \writechord{B} \writechord{f7} \writechord{f7}
    \end{intro}
    \begin{chorus}
        ^{c}Byłaś serca b^{B}iciem ^{f7} \\
        ^{c}Wiosną, zimą, ż^{B}yciem ^{f7} \\
        ^{c}Marzeń moich e^{B}chem ^{f7} \\
        ^{c}Winem, wiatrem, śmi^{B}echem ^{f7}
    \end{chorus}
    \begin{verse}
        Ostatnio w ^{c}mieście mym tramwaje ^{B}po północy b^{f7}łądzą \\
        Rozkładem n^{c}ocnych tras piekielne ^{B}jakieś moce rz^{f7}ądzą \\
        Nie wiedzieć ^{c}czemu wciąż rozkł^{B}ady jazdy ta^{f7}k zmieniają \\
        Że prawie k^{c}ażdy tramwaj ^{B}pod twym oknem ^{f7}nocą staje
    \end{verse}
      \begin{chorus}
        Byłaś serca biciem\ldots
    \end{chorus}
    \begin{verse}
        Ostatnio słońca mniej, ostatnio noce bardziej ciemne \\
        Już nawet księżyc drań o tobie nie chce gadać ze mną \\
        W kieszeni grosze dwa, w kieszeni na dwa szczęścia grosze \\
        W tym jednak losu żart, że ja obydwa grosze noszę
    \end{verse}
    \begin{chorus}
        Byłaś serca biciem\ldots
    \end{chorus}
    \begin{interlude}
        ^{Ab}Ktoś p^{f7}ytał jak się m^{Eb}asz, j^{c}ak się czujesz \\
        ^{Ab}Ktoś, z k^{f7}im rok w wojnę g^{Eb}rasz wycze^*{g}ku ^{C}je \\
        ^{Ab}Ktoś, k^{f7}to nocami, u^{Eb}licami, t^{c}ramwajami \\
        ^{Ab}Pod twe okno m^{f7}knie, gdzie spo^*{F}ty ^{Gb}ka ^{G}mnie
    \end{interlude}
    \begin{chorus}
        Byłaś serca biciem\ldots
    \end{chorus}
    \begin{interlude}
        Ktoś pytał jak się masz\ldots
    \end{interlude}
\end{song}

