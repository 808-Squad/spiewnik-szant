\newpage
\begin{song}{title={Dziesięć w skali Beauforta}, music={Krzysztof Klenczon}}
    \begin{multicols}{2}
    \begin{verse}
        Ko^{g}łysał nas zac^{c}hodni wiatr \\
        ^{D7}Brzeg gdzieś za rufą z^{g}ostał \\
        I n^{c}agle ktoś jak p^{g}apier zbladł \\
        ^{A7}Sztorm idzie, panie ^{D}bosman
    \end{verse}
    \begin{chorus}
        A ^*{Eb}bo sm^{B}an tylko ^{Eb}zapiął pł^{B}aszcz \\
        I z^{Eb}aklął: \say{^{D7}Ech, do cz^{g}orta \\
        Nie da^{Eb}ję ła^{F}jbie ż^{B}adnych s^{g}zans} \\
        ^{g}Dziesięć w ^{c}skali ^*{D7}Beau ^{g}forta
    \end{chorus}
    \vfill\null\columnbreak{}
    \begin{verse}
        Z zasłony ołowianych chmur \\
        Ulewa spadła nagle \\
        Rzucało nami w górę, w dół \\
        I fala zmyła żagle
    \end{verse}
    \begin{chorus}
        A bosman tylko zapiął płaszcz\ldots
    \end{chorus}
    \begin{verse}
        Gdzie został ciepły, cichy kąt \\
        I brzegu kształt znajomy \\
        Zasnuły mgły daleki ląd \\
        Dokładnie, z każdej strony
    \end{verse}
    \begin{chorus}
        A bosman tylko zapiął płaszcz\ldots
    \end{chorus}
    \begin{verse}
        O pokład znów uderzył deszcz \\
        I padał już do rana \\
        Piekielnie ciężki to był rejs \\
        Szczególnie dla bosmana
    \end{verse}
    \begin{chorus}
        A bosman tylko zapiął płaszcz \\
        I zaklął: \say{Ech, do czorta \\
        Przedziwne czasem sny się ma} \\
        Dziesięć w skali Beauforta \medskip \\
        Dziesięć w skali Beauforta \\
        Dziesięć w skali Beauforta
    \end{chorus}
    \end{multicols}
\end{song}

