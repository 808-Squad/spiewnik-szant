\newpage
\begin{song}{title={La Valette}, interpret={Perły i Łotry}, music={tradycyjna (Le Loup, le Renard et la Belette)}}
    \begin{intro}
        Hej, ho! brakuje oregano! \smallskip \\
        \writechord{B} \writechord{C}
    \end{intro}
    \begin{multicols}{2}
    \begin{chorus}
        ^{g}Pierwszy raz, kiedym stanął w La Valette \\
        ^{B}Bitwy smak poczuły ^{C}usta me \smallskip \\
        Pierwszy raz, kiedym stanął w La Valette \\
        Bitwy smak poczuły usta me \smallskip \\
        Na ^{g}prawo bić, na lewo lać \\
        ^{B}Na kolana Anglio, ^{C}dzisiaj Francji czas \smallskip \\
        Na prawo bić, na lewo lać \\
        Na kolana Anglio, dzisiaj Francji czas $\times 2$
    \end{chorus}
    \smallskip
    \begin{verse}
        Z ^{g}morza widzę Malty brzeg \\
        ^{C}Zbliża się wyzwanie \\
        W ^{g}gardłach armat lont już wrze \\
        ^{C}Kończyć ładowanie! \smallskip \\
        ^{g}Niespokojna żagli biel \\
        ^{C}Trzy kolowy flagi \\
        ^{g}Dalej bracia, równać cel \\
        ^{C}Król zostanie nagi!
    \end{verse}
    \smallskip
    \begin{chorus}
        Pierwszy raz, kiedym stanął w La Valette\ldots $\times 2$
    \end{chorus}
    \vfill\null\columnbreak{}
    \begin{verse}
        Zaraz padnie pierwszy strzał \\
        Już się wiara zbroi \\
        Każdy z nas tej bitwy chciał \\
        Dalej, bić Angoli! \smallskip \\
        Kapitana groźny wzrok \\
        Gromki krzyk załogi \\
        Zaraz zrobi pierwszy krok \\
        \textit{« Vive la France ! »} przeraża wrogów
    \end{verse}
    \begin{interlude}
        \textit{(akordy jak w refrenie)} \\
        \textit{(solo na banjo, jeżeli ktoś ma banjo)}
    \end{interlude}
    \begin{chorus}
        Pierwszy raz, kiedym stanął w La Valette\ldots
    \end{chorus}
    \begin{verse}
        Zapach prochu, błyski szpad \\
        W słońcu lśni fregata \\
        Pierwszy pocisk obok spadł \\
        Dalej, bić psubrata! \smallskip \\
        Dla Francuzów idziem w bój \\
        Za wolności śladem \\
        Trzy kolory to nasz strój \\
        Trzęsie Angol zadem
    \end{verse}
    \begin{chorus}
        Pierwszy raz, kiedym stanął w La Valette\ldots $\times 4$
    \end{chorus}
    \begin{outro}
        Hej, ho! brakuje oregano! \\
        \writechord{B} \writechord{C} \writechord{g}
    \end{outro}
    \vfill\null
    \end{multicols}
\end{song}

